\documentclass[a4paper,12pt]{article}
\usepackage[utf8]{inputenc}
\usepackage{amsmath,amsfonts,amssymb}
\usepackage{graphicx}
\usepackage{float}
\usepackage{caption}
\usepackage{listings}
\usepackage{color}
\usepackage{hyperref}
\usepackage{geometry}
\geometry{left=2.5cm, right=2.5cm, top=2.5cm, bottom=2.5cm}

% Definición de colores para el código
\definecolor{codegray}{gray}{0.9}
\definecolor{codeblue}{rgb}{0,0,0.6}

% Configuración para el entorno de listados
\lstset{
    backgroundcolor=\color{codegray},
    basicstyle=\ttfamily\small,
    keywordstyle=\color{blue},
    stringstyle=\color{codeblue},
    commentstyle=\color{gray},
    showstringspaces=false,
    numbers=left,
    numberstyle=\tiny\color{gray},
    stepnumber=1,
    frame=single,
    tabsize=4,
    breaklines=true
}

\title{Informe de Simulación: Mantenimiento de Coches}
\author{Eisler F. Valles Rodríguez, Rafael Acosta Márquez}

\date{Junio 2024}

\begin{document}

    \maketitle

    \tableofcontents

    \newpage


    \section{Introducción}\label{sec:introduccion}

    \subsection{Breve Descripción del Proyecto}\label{subsec:breve-descripcion-del-proyecto}
    Este proyecto se centra en la simulación del proceso de mantenimiento de coches en una compañía de alquiler.
    Utilizamos la simulación basada en eventos discretos para modelar un taller de mantenimiento que opera las 24 horas del día, evaluando la viabilidad de cambiar a un nuevo servicio de mantenimiento más rápido pero más costoso.

    \subsection{Objetivos y Metas}\label{subsec:objetivos-y-metas}
    \begin{itemize}
        \item Evaluar el costo total del mantenimiento actual y del tiempo de inactividad de los coches.
        \item Analizar la viabilidad económica de cambiar a un nuevo servicio de mantenimiento que reduce el tiempo de servicio a costa de un mayor costo diario.
        \item Determinar el incremento máximo en el costo del nuevo servicio de mantenimiento que sigue siendo rentable para la compañía.
    \end{itemize}

    \subsection{Sistema a Simular y Variables de Interés}\label{subsec:sistema-a-simular-y-variables-de-interes}
    El sistema específico a simular es un taller de mantenimiento de coches con las siguientes características:
    \begin{itemize}
        \item \textbf{Capacidad del taller}: El taller puede atender un coche a la vez.
        \item \textbf{Horario de operación}: El taller opera 24 horas al día.
        \item \textbf{Tasa de llegada de coches}: Los coches llegan al taller con una media de 3 coches por día.
        \item \textbf{Tiempo de mantenimiento}: El servicio de mantenimiento sigue una distribución exponencial con una media de 7 horas.
        \item \textbf{Costo de mantenimiento}: El servicio actual cuesta 375 euros por día.
        \item \textbf{Costo de inactividad}: Cada coche parado sin poder ser alquilado cuesta a la compañía 25 euros por día.
        \item \textbf{Nuevo servicio de mantenimiento}: Puede reducir el tiempo de mantenimiento a una media de 5 horas, pero con un costo adicional.
    \end{itemize}

    \subsection{Variables que Describen el Problema}\label{subsec:variables-que-describen-el-problema}
    \begin{itemize}
        \item \textbf{Tasa de llegada de coches} ($\lambda$): 3 coches por día.
        \item \textbf{Tiempo de mantenimiento} ($\mu$): Media de 7 horas actualmente, con la opción de reducirse a 5 horas.
        \item \textbf{Costo de mantenimiento diario}: 375 euros.
        \item \textbf{Costo de inactividad por coche}: 25 euros por día.
        \item \textbf{Capacidad del taller}: 1 coche.
        \item \textbf{Costo adicional permitido para el nuevo servicio}: A determinar.
    \end{itemize}


    \section{Detalles de Implementación}\label{sec:detalles-de-implementacion}
    \subsection{Pasos Seguidos para la Implementación}\label{subsec:pasos-seguidos-para-la-implementacion}

\begin{enumerate}
    \item \textbf{Importación de Librerías}: Se utilizaron \texttt{simpy} para la simulación, así como \texttt{matplotlib} para la visualización de datos.

    \begin{lstlisting}[language=Python, caption=Importación de Librerías,label={lst:lstlisting}]
import matplotlib.pyplot as plt
import simpy
    \end{lstlisting}

    \item \textbf{Modelado del Proceso de Mantenimiento}: Se definió una clase Workshop que modela el proceso de mantenimiento de un taller.

    \begin{lstlisting}[language=Python, caption=Clase Workshop,label={lst:lstlisting2}]
class Workshop:

    def __init__(self, env, average_repair_time):
        self.env = env
        self.average_repair_time = average_repair_time
        self.resource = simpy.Resource(env, capacity=1)

    def repair_car(self):
        repair_time = random.expovariate(1 / self.average_repair_time)
        yield self.env.timeout(repair_time)
        reparation_times.append(repair_time)
        return repair_time

    \end{lstlisting}

    \pagebreak
    \item \textbf{Simulación Continua del Taller}: Se creó una función que modela la llegada de coches al taller y el inicio del proceso de mantenimiento.

    \begin{lstlisting}[language=Python, caption=Función de simulación de llegada de autos,label={lst:lstlisting3}]
        
def car(env, name, workshop, results):
    arrival_time = env.now

    with workshop.resource.request() as req:
        yield req
        start_repair_time = env.now
        _ = yield env.process(workshop.repair_car())
        end_repair_time = env.now
        time_stopped = end_repair_time - arrival_time
        results.append(
            {
                "name": name,
                "arrival_time": arrival_time,
                "start_repair_time": start_repair_time,
                "end_repair_time": end_repair_time,
                "time_stopped": time_stopped,
            }
        )


    \end{lstlisting}

    \item \textbf{Ejecución de la Simulación}: Se implementó una función principal que corre la simulación durante un periodo especificado y calcula el costo total total del tiempo parado, todo esto recibiendo el delay entre las llegadas de los autos

    \begin{lstlisting}[language=Python, caption=Función Principal de Simulación,label={lst:lstlisting6}]
def setup(env, average_repair_time, results, car_arrivals_delay):

    workshop = Workshop(env, average_repair_time)

    car_count = itertools.count()
    idx = 1
    while True:
        yield env.timeout(car_arrivals_delay[idx])
        env.process(car(env, f"Car {next(car_count)}", workshop, results))
        idx +=1
    \end{lstlisting}

    \pagebreak
    \item \textbf{Simulación de llegada de autos}: Para mantener la consistencia entre cada simulación con respecto al tiempo de llegada de los autos, se generó una lista con estos tiempos la cual es usada tanto para el sistema actual como para el nuevo

    \begin{lstlisting}[language=Python, caption=Simulación de delay entre llegada de autos,label={lst:lstlisting4}]
def simulate_car_arrival(env, car_arrival_de):

    while True:
        arrival_time = random.expovariate(1 / AVERAGE_ARRIVAL_TIME)
        yield env.timeout(arrival_time)
        car_arrival_de.append(arrival_time)


def run_car_arrival():
    env = simpy.Environment()
    car_arrival_delay = []
    env.process(simulate_car_arrival(env, car_arrival_delay))
    env.run(until=SIMULATION_TIME)

    # add a dummy value that never gets processed
    car_arrival_delay.append(1000)

    return car_arrival_delay

    \end{lstlisting}
    \pagebreak
    \item \textbf{Visualización de Resultados}: Se graficaron los tiempos totales de inactividad para el taller actual y el nuevo servicio. Así como el máximo incremento permitido en el coste del nuevo taller

    \begin{lstlisting}[language=Python, caption=Visualización de Resultados,label={lst:lstlisting7}]
        
def run_sims(simulations: int):
    results = []
    initial_stop_hours = []
    new_stop_hours = []
    initial_total_costs = []
    new_total_costs = []
    for i in range(simulations):

        car_arrivals_delay = run_car_arrival()

        initial_cost, logs = run_simulation(
            AVERAGE_REPAIR_TIME, SIMULATION_TIME, car_arrivals_delay
        )

        # Calculate the total stop hours
        stop_hours = sum(i["time_stopped"] for i in logs)
        initial_stop_hours.append(stop_hours)

        # Calculate the initial total cost
        total_initial_cost = initial_cost + TOTAL_DAYS * MAINTENANCE_COST
        initial_total_costs.append(total_initial_cost)

        new_cost, new_logs = run_simulation(
            NEW_AVERAGE_REPAIR_TIME, SIMULATION_TIME, car_arrivals_delay
        )

        # Calculate the new total stop hours
        stop_hours = sum(i["time_stopped"] for i in new_logs)
        new_stop_hours.append(stop_hours)

        # Calculate the max allowed daily cost increase
        max_allowed_daily_cost_increase = (total_initial_cost - new_cost) / TOTAL_DAYS

        # Calculate the new total cost
        new_total_cost = new_cost + TOTAL_DAYS * MAINTENANCE_COST
        new_total_costs.append(new_total_cost)

        results.append(max_allowed_daily_cost_increase - MAINTENANCE_COST)

    mean = sum(results) / simulations
    print(f"Average max allowed daily cost increase: {mean}")

    plot_costs(initial_total_costs, new_total_costs, simulations)
    plot_stop_hours(initial_stop_hours, new_stop_hours, simulations)
    plot_max_allowed_cost_increase(mean, results, simulations)


    \end{lstlisting}
\end{enumerate}


    \section{Resultados y Experimentos}\label{sec:resultados-y-experimentos}
    \subsection{Hallazgos de la Simulación}\label{subsec:hallazgos-de-la-simulacion}

Se realizaron simulaciones para ambos escenarios de mantenimiento: el actual con una media de 7 horas por coche y el nuevo con una media de 5 horas.
Ambos escenarios fueron simulados 1000 veces para evaluar y se encontró la media del costo máximo adicional del nuevo servicio para seguir siendo rentable.
Los resultados indican que el incremento máximo en el costo del nuevo servicio de mantenimiento para seguir siendo rentable es de aproximadamente 125 euros por día .

\subsection{Interpretación de los Resultados}\label{subsec:interpretacion-de-los-resultados}

Los resultados muestran que, aunque el nuevo servicio de mantenimiento reduce significativamente el tiempo de inactividad de los coches, sólo es económicamente viable si el costo diario adicional no supera los 125 euros.
Esto se debe a que la reducción en los costos de inactividad debe compensar el aumento en el costo del servicio, lo cual si tenemos en cuenta que la media de coches diarios se mantiene y la diferencia
en el tiempo de mantenimiento es de 2 horas no debería permitir un gran incremento el coste

\subsection{Hipótesis Extraídas de los Resultados}\label{subsec:hipotesis-extraidas-de-los-resultados}

\begin{itemize}
    \item Reducir el tiempo de mantenimiento puede ser económicamente ventajoso si el costo adicional del nuevo servicio no supera un umbral específico.
    \item La capacidad del taller (un coche a la vez) y el costo de inactividad son factores críticos en la evaluación de la viabilidad de cambiar el servicio de mantenimiento.
\end{itemize}

\subsection{Experimentos Realizados para Validar las Hipótesis}\label{subsec:experimentos-realizados-para-validar-las-hipotesis}

Se realizaron múltiples ejecuciones de la simulación(1000) variando la media del tiempo de mantenimiento del nuevo servicio para determinar el costo adicional máximo que sigue siendo rentable.

\subsection{Análisis Estadístico de la Simulación}\label{subsec:analisis-estadistico-de-la-simulacion}

El análisis estadístico es crucial para entender la variabilidad en los tiempos de espera y de mantenimiento, así como su impacto en los costos.
Se utilizaron histogramas para comparar las distribuciones de los tiempos de espera y de mantenimiento(tiempo total de inactividad)en ambos escenarios.


%\pagebreak
\begin{figure}[H]
    \centering
    \includegraphics[width=0.9\textwidth]{img/stop_hours}
    \caption{Distribución de Tiempos de totales de inactividad para el Taller Actual y el Nuevo Servicio}
    \label{fig:-tiempo_total_de_inactividad}

\end{figure}


Como se puede observar en la Figura~\ref{fig:-tiempo_total_de_inactividad}, el nuevo servicio reduce significativamente el tiempo total de inactividad de los coches en comparación con el taller actual.


\begin{figure}[H]
    \centering
    \includegraphics[width=0.9\textwidth]{img/costs}
    \caption{Costos del Taller Actual y el Nuevo Servicio usando el mismo costo por día}
    \label{fig:-costos-totales}
\end{figure}

En la imagen~\ref{fig:-costos-totales} se puede observar que el nuevo servicio de mantenimiento es muy rentable al actual si el costo adicional es 0.

\begin{figure}[H]
    \centering
    \begin{minipage}{0.7\textwidth}
        \centering
        \includegraphics[width=0.9\textwidth]{img/initial_total_costs}
        \caption{Costos del Taller inicial}
        \label{fig:-costo-total-inicial}
    \end{minipage}
    \begin{minipage}{0.7\textwidth}
        \centering
        \includegraphics[width=0.9\textwidth]{img/cost_with_price_increase}
        \caption{Costos del Taller Actual}
        \label{fig:-nuevo-costo-total}
    \end{minipage}

\end{figure}

En la imagen~\ref{fig:-costo-total-inicial} y ~\ref{fig:-nuevo-costo-total} se puede observar que aun aumentando el costo en el nuevo sistema las dos gráficas muestran los mismos costes

\subsection{Análisis de Parada de la Simulación}\label{subsec:analisis-de-parada-de-la-simulacion}


La simulación se corrió durante un periodo de 365 días para capturar adecuadamente la variabilidad en los tiempos de llegada y de mantenimiento.
Además esta misma se repitió 1000 veces para obtener una media más precisa de los costos.


    \section{Modelo Matemático}\label{sec:modelo-matematico}

    \subsection{Descripción del Modelo}\label{subsec:descripcion-del-modelo}

    El modelo matemático utilizado se basa en distribuciones exponenciales para la llegada de coches y el tiempo de mantenimiento:

    \begin{itemize}
        \item \textbf{Distribución Exponencial de Llegadas}: La llegada de coches se modela con una tasa $\lambda = 3$ coches por día.
        \item \textbf{Distribución Exponencial de Servicio}: El tiempo de mantenimiento se modela con una tasa $\mu = \frac{1}{7}$ horas para el servicio actual y $\mu = \frac{1}{5}$ horas para el nuevo servicio.
    \end{itemize}

    La función de distribución exponencial es:

    \[
        P(T \leq t) = 1 - e^{-\lambda t}
    \]

    \subsection{Supuestos y Restricciones}\label{subsec:supuestos-y-restricciones}

    \begin{itemize}
        \item \textbf{Llegadas Independientes}: Se asume que las llegadas de coches son independientes entre sí.
        \item \textbf{Capacidad Constante}: El taller solo puede atender un coche a la vez.
        \item \textbf{Costo Constante por Inactividad}: El costo diario por tener un coche parado es constante a 25 euros.
        \item \textbf{Operación Continua}: El taller opera de manera continua las 24 horas.
    \end{itemize}

    \subsection{Comparación de Resultados con los Datos Experimentales}\label{subsec:comparacion-de-resultados-con-los-datos-experimentales}

    Los resultados experimentales obtenidos a través de la simulación muestran una buena coherencia con los modelos matemáticos esperados para distribuciones exponenciales.
    La evaluación del incremento máximo en el costo es consistente con la teoría económica de costos de oportunidad, donde la reducción en el tiempo de inactividad compensa el costo adicional del nuevo servicio.


    \section{Conclusión}\label{sec:conclusion}

    El análisis sugiere que la adopción del nuevo servicio de mantenimiento es viable hasta un incremento de aproximadamente 125 euros por día en el costo.
    El enfoque de simulación y el modelo matemático utilizado proporcionan una base sólida para la toma de decisiones en la gestión del servicio de mantenimiento de coches de la compañía.
    La simulación muestra cómo el cambio en la política de mantenimiento puede optimizar los costos operativos y reducir el tiempo de inactividad de los coches.

    \bibliographystyle{plain}
    \bibliography{bibliografia}


\end{document}
